\documentclass[14pt]{extreport}
\usepackage{geometry} % see geometry.pdf on how to lay out the page. There's lots.
\usepackage{amsmath}
\usepackage{amssymb}
\usepackage{fancyhdr}
\usepackage{pgfplots}
\usepackage{setspace}
\pgfplotsset{compat=1.17} 
\geometry{a4paper} % or letter or a5paper or ... etc
% \geometry{landscape} % rotated page geometry


\title{Discrete Mathematics Notes}
\author{Jonathan Parlett}

\begin{document}
\paragraph{1.0} \textbf{Sets}\\\\

Two sets are equal if they contain the same elements. $A = B$ means $\forall x[(x \in A \Rightarrow x \in B) \land (x \in B \Rightarrow x \in A)]$.\\

There are a few ways to describe a set.\\
1. List the elements $S = {2, 4 ,6}$.\\
2. Describe the elements recursively. $1 \in S$. If $k \in S$ then $(k+1) \in S$. The set of all integers
3. Describe a property $P$ of that characterizes the sets elements $S =$ {x | x is a positive even integer}

The notation for a set $S$ whose elements are characterized as having a property $P$ is ${x | P(x)}$. Where $P$ is a unary predicate (an expression taking one argument).\\

\textbf{Some special set notations}\\
$\mathbb{N}$ = Natural Numbers.\\
$\mathbb{Z}$ = Integers.\\
$\mathbb{Q}$ = Rational Numbers.\\
$\mathbb{R}$ = Real Numbers.\\
$\mathbb{C}$ = Complex Numbers.\\
$\varnothing = $ The empty/null set\\\\

A set $A$ is said to be a subset of a set $B$ is every member of $A$ is also a member of $B$. This is stated as $ A \subseteq B$ when $A$ possibly equals $B$.
$A$ is said to be a proper subset of $A$ if all elements of $A$ are in $B$ but not all elements of $B$ are in $A$ stated as $ A \subset B$.\\

$A \subseteq B = \forall x(x \in A \Rightarrow x \in B)$ notice you do not have the reverse that would state $A=B$.\\\\

Notice $x \in \varnothing$ is always false since nothing is in the empty set. Conversely $\varnothing \subset B \land \varnothing \subseteq B$ is always true. This just means the empty set is subset of everything.\\\\

Remeber to prove equivalence you must prove implication both ways. To show that two sets $A$ and $B$ are equal you must show that ($A \subseteq B) \land (B \subseteq A)$.\\

The powerset of a set $S$, denoted $\wp(S)$ is the set of all subsets of $S$. $S = {0,1}$, $\wp(S) = {\varnothing, {0}, {1}, {0, 1}}$.\\\\

\textbf{Binary Operation Definition: }$\bullet$ is a binary operation on a set $S$ if for every ordered pair $(x, y)$ of elements of $S$, $x \bullet y$ exists is unique and is a member of $S$.\\
In other words $S$ must be closed under the binary operation $\bullet$.\\\\

\textbf{Union and Intersection: } Let $A,B \in \wp(S)$. The union of $A$ and $B$, denoted by $A \cup B$, is {$x | x \in A \lor x \in B$}. The intersection of $A$ and $B$, denoted $A \cap B$, is {$x | (x \in A) \land (x \in B)$}.\\
Union of two sets is the set of all elements in either, and intersection is set of just the common elements of the two.\\\\

\textbf{Complement of a Set: }For set $A \in \wp(S)$, the complement of $A$, denoted $A'$, is {$x | (x \in S) \land (x \notin A)$}.\\
In other words the inverse of $A$ is the set of everything except the elements of $A$.\\\\

\textbf{Set Difference:} $A-B = ${$x | (x \in A) \land (x \notin B)$}. This is equivalent to $A-B = {x | (x \in A) \land (x \in B')}$.\\
In english $A-B$ is the set of all elements in $A$ except those in $B$.\\\\

$A$ and $B$ are said to be disjoint if $A \cap B = \varnothing$ (if they have no common elements). Thus $A - B$ and $B - A$ are disjoint sets.\\\\

\textbf{The Cartesian Product: }Let $A$ and $B$ be subsets of $S$. The Cartesian product (cross product) of $A$ and $B$, denoted $A \times B$, is defined by\\
$A \times B = { (x,y) | (x \in A) \land (y \in B)}$.\\
The cross product is not a binary operation because its result is generally not a subset of $S$.\\\\

\textbf{Basic Set Identities}\\
1a. Commutative $A \cup B = B \cup A$ \\
1b. $A \cap B = B \cap A$.\\
2a. Associative $(A \cup B) \cup C = A \cup (B \cup C)$ \\
2b. $(A \cap B) \cap C = A \cap (B \cap C)$.\\
3a. Distributive $A \cup (B \cap C) = (A \cup B) \cap (A \cup C)$ \\
3b. $A \cap (B \cup C) = (A \cap B) \cup (A \cap C)$.\\
4a. Identity $A \cup \varnothing = A$ \\
4b.$ A \cap S = A$.\\
5a. Complement $ A \cup A' = S$ \\
5b. $A \cap A' = \varnothing$.\\\\

The \textbf{dual} for a set identity is obtained by interchanging $\cup$ and $\cap$ and intechanging $\varnothing$ and $S$. Anytime we have proved a set identity we have also proved
its dual.\\\\

The number of elements in a finite set is the cardinality of the set. For a set $S = {s_1, s_2, \cdots, s_k}$ then the cardinality $|S|$ is $k$. $|S| = k$.\\
If the set is infinite we may still be able to select the first second and third elements so that $s_1, s_2, \cdots$ represents all elements of the set.\\
This kind of infinite set is said to be denumerable or countable.

\paragraph{2.0} \textbf{Counting}\\\\

\textbf{Multiplication Principle}: If there are $n_1$ possible outcomes for a first even and $n_2$ possible outcomes for a second event, there are $n_1 \times n_2$ possible outcomes for the sequence of the two events.\\
This principle can be extended to an arbitrary number of elements. Take the $4$ digits of a phone number example. How many ways are there to construct the last $4$ digits of a phone if each must be unique.
Given that there are $10$ digits this means $10$ ways to choose the first, $9$ ways to choose the second, $8$ for the $3$rd, $7$ for the $4$th = $10 * 9 * 8 * 7 = 5040$. If repitions are allowed then there are ten
ways to choose each digit. $10*10*10*10 = 10000$.\\
This leads us the cross product of two sets. $ |A \times B| = |A| * |B|$.\\\\

\textbf{Addition Principle}: If $A$ and $B$ are disjoint events with $n_1$ and $n_2$ possible outcomes, respectively, then the total number of possible outcomes for event $A$ or $B$ is $n_1 + n_2$.\\
This principle can also be extended to the case of any finite number of disjoint sets.\\
This principle speaks about when two events are unrelated/exclusive to each other. If you want to buy a car or a truck and there are 20 cars and 3 trucks availble then obviously you only have 23 possible outcomes because you are not buying a car and a truck.\\
$A \cup B| = |A| + |B|$. This describes the addition principle. Recall that $A$ and $B$ must be disjoint sets.\\\\

If $A$ and $B$ are finite sets\\
$|A - B| = |A| - |A \cap B|$ and $|A - B| = |A| - |B|$ if $B \subseteq A$.\\\\
\textbf{What is a identity:} An equation that is true no matter what values are chosen $a * 0 = 0$ for all $a$.\\\\

\paragraph{2.1} \textbf{Principle of Inclusion and Exclusion}\\\\
$|A \cup B| = |A| + |B| - |A \cap B|$.\\\\

\textbf{Inclusion Exclusion Example}\\
A pollster queries 35 voters all of whom support refereendum 1, referendum 1, referendum 2m or both, and finds that 14 voters support referendum 1  and 26 support referendum 2.
How many voters support both?\\
Let $A$ be the set of voters supporting referendum $1$ and $B$ be the set of voters referendum 2, then we know that\\
$|A \cup B| = 35$  $|A| = 14$  $|B| = 26$\\
From equation 2,\\
$|A \cup B| = |A| + |B| - |A \cap B|$\\
$35 = 14 + 26 -|A \cap B|$\\
$|A \cap B| = 14 + 26 - 35 = 5$\\
so 5 voters support both.\\\\

The exclusion principle can be extended to 3 sets easily. $|A \cup B \cup C| = |A| + |B| + |C| - |A \cap B| - |A \cap C| -|B \cap C| + |A \cap B \cap C|$.\\\\

So the principle of  inclusion and exclusion can be stated formally as\\
Given the finite sets $A_1, \cdots, A_n$ $n \ge 2$, then\\
$|A_1 \cup \cdots \cup A_n| = \sum_{1 \le i \le n} |A_i| - \sum_{1 \le i < j \le n} |A_i \cap A_j| + \sum_{1 \le i < j < k \le n} | A_i \cap A_jj \cap A_k| - \cdots + (-1)^{n+1} |A_1 \cap \cdots \cap A_n|$\\\\

\textbf{Pigeonhole Principle}: If more than $k$ items are placed into $k$ bins, then at least 1 bin contains more than 1 item.\\\\

\paragraph{2.2} \textbf{Graph Theory}\\
\\
A tree is a visual representation of data items and the connections between some of them (implies not necessairlily all). It is a special case of a more general structure called
a graph.\\
\\
\textbf{Graph Definition Informal:} A graph is a set of nodes (vertices) and a set of arcs (edges) such that each arc connects two nodes.\\
\\
\textbf{Graph Definition Formal:} A graph is an ordered triple $(N, A, g)$ where\\
$N = $a nonempty set of nodes\\
$A$ = a set of arcs (edges)\\
$g$ = a function associating with each arc $a$ an unordered pair $x-y$ of nodes called the endpoints of $a$.\\


\textbf{Directed Graph Definition:} A directed graph (digraph) is an ordered triple $(N,A,g)$ where\\
$N = $a nonempty set of nodes\\
$A$ = a set of arcs (edges)\\
$g$ = a function associating with each arc $a$ an ordered pair $(x,y)$ of where $x$ is the initial point and $y$ is the terminal point of $a$.\\

\textbf{Labled Graph:} Graph where nodes carry identifing information.\\
\textbf{Weighted Graph:} Graph each edge/arc has a numerical weight.\\
\\
\textbf{Parallel Arcs:} Two arcs with the same endpoints are parallel arcs. \\
\textbf{Isolated NOde:} an isolated node is adjacent to no other node.\\
\textbf{Simple Graph:} one with no loops or parallel arcs.\\
\textbf{Degree:} the degree of a node is the number of arcs that end at that node.\\
\\
Each arc has a unique pair of endpoints if $g$ is a one to one function.\\
\textbf{Complete Graph:} One in which any two distinct nodes are adjacent.\\
\textbf{subgraph:} a graph that is a set of nodes and a set of arcs that are subsets of the original node set and arc set.\\
\textbf{path:} a sequence of nodes starting at one node and ending at another (possibly the same) node. $n_0,a_0,n_1,a_1, \cdots, n_{k-1},n_k$\\
\textbf{path length:} number of arcs in the path.\\
\textbf{connected graph:} a graph in which there is a path from any node to anyother node.\\
\textbf{cycle:} a cycle is path from a node $n_0$ to a node $n_0$. Basically a circular path.\\
\textbf{adjacent:} two nodes are adjacent if their endpoints are associated with an arc.\\
\textbf{bipartite complete graph:} A graph such that its nodes can be partitioned into two disjoint nonempty sets $N_1$ and $N_2$ such that two nodes $x$ and $y$
are adjacent iff $x \in N_1 \land y \in N_2$. If $|N_1| = m$ and $|N_2| = n$, such a graph is denoted $K_{m,n}$.\\
\textbf{directed graph path:} a sequence $n0,a_0, \cdots, a_{k-1},n_k$ where each $i,n$ is the initial point and $n_{i+1}$ is the terminal point of $a_i$. If a path
exists between node that a either node is said to be reachable by the other.\\
\textbf{isomorphic structures/graphs:} structures that are the same except for relabling are called isomorphic structures. If two structures are isomorphic then there because
a one-to-one and onto mapping between the elements of the two structures. Basically if graph $G$ has a an arc with endpoints $x-y$ and $G'$ also has an arc with the same endpoints
for all endpoints and arcs then the graphs $G$ and $G'$ can be called isomorphic.\\

\textbf{isomorphic graph formal def:} Two graphs $(N_1,A_1,G_1)$ and $(N_2,A_2,G_2)$ are isomorphic if there are bijections $f_1: N_1 \to N_2$ and $f_2: A_1 \to A_2$
such that for each arc $a \in A_1, g_1(a)=x-y$ if and only if $g_2[f_2(a)]=f_1(x)-f(y)$.\\
\\
\textbf{simple graph isomorphism:} Two simple graphs $(N_1,A_1,G_1)$ and $(N_2,A_2,G_2)$ are isomorphic if there is a bijection $f:N_1 \to N_2$ such that for any
nodes $n_i$ and $n_j$ of $N_1$, $n_i$ and $n_j$ are adjacent if and only if $f(n_i)$ and $f(n_j)$ are adjacent. The function $f$ is called an isomorphism from graph 1 to graph 2.\\
\\
\textbf{Reasons Graphs are not isomorphic}
1. One graph has more nodes than the other.
2. One graph has more arcs than the other.
3. One graph has parallel arcs and the other does not.
4. One graph has a loop and the other does not.
5. One graph has a node of degree k and the other does not.
6. One graph is connected and the other is not.
7. One graph has a cycle and the other does not.

\textbf{planar graph:} one that can be represented so that its arcs intersect only at node (when drawn on a piece of paper).\\

\textbf{Euler's formula for planar graphs:} $n-a+r=2$ where $n$= number of nodes, $a$=number of arcs, and $r$=regions in the graph.\\
If $n \ge 3$, then $a \le 3n -$\\
If $n \ge 3$ and there are no cycles of length $3$, then $a \le 2n -4$.\\
\section{2.3} \textbf{Combinatorics}\\
\\
\textbf{Permutation}: ordered arragement of objects.\\
The number of $r$ distince Objects chosen from $n$ distinct objects is denoted $P(n,r)$.\\
$P(n,r) = \frac{n!}{(n-r)!}$\\
\\
By definition $P(n,0)=1 \land P(n,1)=n$.\\
\\
Example: Permutations of 3 objects $a,b,c$ = $P(3,3)=6$.\\
\\
\subsection{Combinations}
When we want to select $r$ objects from a set of $n$, but we don't care how they are arranged then we are counting combinations.\\
$C(n,r)$ = combinations = $\frac{P(n,r)}{r!} = \frac{n!}{r!(n-r)!} = \binom{n}{r}$\\

The number of distinct permutations of $n$ objects in which $n_1 ,\cdots, n_k$ are indistinguable from one another is $\frac{n!}{(n_1!) \cdots (n_k!)}$\\

The number of permutations of $r$ objects out of $n$ distinct objects with repetition allowed is $n^r$. Because repeition is allowed there
are $n$ ways to choose the first object and also $n$ ways to choose the second and third and fourth, etc.\\

Ways to select $r$ our $n$ distinct objects with repetitions: $C(r+n-1,r)$.\\


\end{document}

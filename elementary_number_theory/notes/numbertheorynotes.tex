\documentclass[14pt]{extreport}
\usepackage{geometry} % see geometry.pdf on how to lay out the page. There's lots.
\usepackage{amsmath}
\usepackage{amssymb}
\usepackage{fancyhdr}
\usepackage{pgfplots}
\usepackage{setspace}
\pgfplotsset{compat=1.17} 
\geometry{a4paper} % or letter or a5paper or ... etc
% \geometry{landscape} % rotated page geometry

% See the ``Article customise'' template for come common customisations

\title{Elementary Number Theory Notes}
\author{Jonathan Parlett}



\begin{document}

\paragraph{1.0} \textbf{Introduction}\\\\

\textbf{Well Ordering Principle}: Every nonempty set $S$ of non-negative integers contains a least element; that is there is osme integer $a$ in $S$ such that $a \le b$ for all $b$'s belonging to $S$\\\\

\textbf{Theorem 1.1:} Archimedian property. If $a$ and $b$ are any positive integers, then there exists a positive integer $n$ such that $na \ge b$.\\\\

\textbf{Theorem 1.2} First Principle of Finite Induction. Let $S$ be the set of positive integers.\\
    (a) The integer 1 belongs to $S$\\
    (b) Whenever the integer $k$ is in $S$, the next integer $k+1$ must also be in $S$\\\\

\textbf{Theorem 1.2} Second Principle of Finite Induction. Let $S$ be the set of positive integers.\\
    (a) The integer 1 belongs to $S$\\
    (b') If $k$ is a positive integer such that $1, 2, \cdots ,k$ for $k \in S$, then $k+1$ must also be in $S$.\\\\


    Thus $S$ is the set of all positive integers.\\\\

  \textbf{Binomial Theorem}\\\\
  $
  \begin{pmatrix}
      n\\
      k\\
  \end{pmatrix}
  = \frac{n!}{k!(n-k)!}$\\\\

  Canceling either $k!$ or $(n-k)!$ yields\\
  
  $\frac{n(n-1) \cdots (k+1)}{(n-k)!}$ or $\frac{n(n-1) \cdots (n-k+1)}{k!}$\\

  If $k=0$ or $k=1$ then we have 
  $\begin{pmatrix}
      n\\
      0\\
  \end{pmatrix} = 
  \begin{pmatrix}
      n\\
      n\\
  \end{pmatrix} = 1
$\\\\

$
\begin{pmatrix}
    n\\
    k\\    
\end{pmatrix}
+
\begin{pmatrix}
    n\\
    k-1\\    
\end{pmatrix}
=
\begin{pmatrix}
    n+1\\
    k\\    
\end{pmatrix}
$\\\\

\textbf{Pascals Triangle}\\
Rows of pascals triangle are built by $(a+b)^n$.\\
$(a+b)^1 = a+b$\\
$(a+b)^2 = a^2 + 2ab + b^2$\\
$(a+b)^3 = a^3 + 3a^2b + 3ab^2 + b^3$\\
$(a+b)^4 = a^4 + 5a^3b + 6a^2b^2 + 4ab^3 + b^4$\\
When $a = b = 1$ the following triangle is built\\
       1 1\\
      1 2 1\\
     1 3 3 1\\
    1 4 6 4 1\\

The binomial expansion takes the form $(a + b)^n = \begin{pmatrix} n \\ 0 \\\end{pmatrix} a^n + \begin{pmatrix} n \\ 1 \\\end{pmatrix} a^{n-1}b + \begin{pmatrix} n \\ 2 \\\end{pmatrix} a^{n-2}b^2 +
\cdots + \begin{pmatrix} n \\ n-1 \\\end{pmatrix}ab^{n-1} + \begin{pmatrix} n \\ n \\\end{pmatrix}b^n$\\\\

or $(a+b)^n = \sum_{k=0}^n \begin{pmatrix} n \\ k \\\end{pmatrix} a^{n-k}b^k$\\\\

\paragraph{1.1} \textbf{Chapter 2}\\

Pythagoreans were pretty weird and attached tons of religious connotations to numbers.\\
The number 1 represents reason\\
The number 2 stood for man\\
The number 3 stood for woman\\
4 stood for justice since it is the first number that is the product of equals\\
5 was for marriage because it formed the union of 2 and 3 (man and woman)\\\\

All sums $1 + \cdots + n$ are actually triangular numbers.\\\\

\textbf{Triangular Numbers}\\
A number is triangular if it is of the form $\frac{n(n+1)}{2}$\\
$n$ is triangular if $8n+1$ is a perfect square.\\
The sum of any consecutive two triangular numbers is a perfect square. It is in fact the nth square. Take $1+1+2=4$ for example. Here $4$ is the square of the second natural number and it is also the sum of the first two summations for $n=1$ and $n=2$ respecitively.\\
If $n$ is triangular then so is $9n+1$, $25n + 3$, and $49n+6$\\

Let $t_n$ denote the $n$th triangular number

$t_n = \begin{pmatrix} n+1 \\ 2 \end{pmatrix}$\\

$t_{n-1} + t_n = n^2$\\

$t_1 + t_2 + t_3 + \cdots  + t_n = \frac{n(n+1)(n+2)}{6}$ the sum of $n$ consecutive triangular numbers is the same as the sum of $n$ consecutive squares because we have previously shown that the sum of the two consecutive triangular numbers is the $n$th square.\\

$\begin{pmatrix} 2 \\ 2 \end{pmatrix} + \cdots \begin{pmatrix} n \\ 2 \end{pmatrix} \begin{pmatrix} (n+1) \\ 3\end{pmatrix}$ \\
This formula is an extension of binomial expansion formula for summations\\

If $t_n$ is a perfect square then $t_{4_n(n+1)}$ is also a perfect square.\\

The difference of two consecutive triangular numbers is a cube.\\

\textbf{Pentagonal Numbers}\\
let $p_n$ = the nth pentagonal number\\

$p_n = \frac{n(3b-1)}{2}$\\

$p_n = t_{n-1} + n^2$\\

$p_n = 3t_{n-1} + n = 2t_{n-1} + t_n$\\

\paragraph{2.2} \textbf{The Division Algorithm}\\

\textbf{Theorem 2.1: Division Algorithm}. Given integers $a$ and $b$ with $b > 0$, there exists unique integers $q$ and $r$ such that\\
$a = qb + r$   $0 \le r < b$\\
$q$ is called the quotient and $r$ the remainder in the division of $a$ by $b$.\\\\

when $b=2$ $r=0$ or $r=1$\\

when $a=2q+0$, $a$ is called even\\

when $a=2q+1$, $a$ is called odd\\

The previous 2 statements are imply that any integer is of the form $2n+1$ or $2n+0$ similary any square is of the form\\\\

${(2q)^2} = 4k$ or $(2q+1)^2 = 4(q^2+q)+1 = 4k+1$ this also implies that any square is has a remainder $0$ or $1$  when divided by $4$\\\\


\textbf{Greatest Common Divisor}\\\\

An int $a$ is said to be divisible by an integer $a \ne 0$, $a|b$ if there exists some integer $c$ such that $b = ac$ we write $a \nmid b$ if $a$ does not divide $b$.\\
$a|b$ means $a$ divides $b$ 

\textbf{Theorem 2.2} \\
1. $a|0, 1|a, a|a$\\
2. $a|1$ iff $a = \pm 1$\\
3. If $a|b$ and $c|d$ then $ac|bd$\\
4. If $a|b$ and $b|c$, then $a|c$\\
5. $a|b$ and $b|a$ iff $a = \pm b$\\
6. If $a|b$ and $b \ne 0$, then $|a| \le |b|$\\
7. If $a|b$ and $a|c$, then $a|(bx+cy)$ for arbitrary $x,y$\\

\textbf{Definition 2.2}: Let $a$ and $b$ be given integers, with at least one different from zero. The GCD of $a$ and $b$, denoted by $gcd(a, b)$ is the positive integer $d$ satisfying the following.\\

1. $d|a$ and $d|b$\\
2. If $c|a$ and $c|b$, then $c \le d$\\

\textbf{Theorem 2.3:} Given integers $a$ and $b$, not both of which are zero, there exists $x$ and $y$ such that $gcd(a, b) = ax + by$.\\

\textbf{Definition 2.3:} Two integers $a$ and $b$, not both of which are zero, said to be relatively prime when $gcd(a,b) = 1$.\\

\textbf{Theorem 2.4:} Let $a$ and $b$ be integers, not both zero. Then $a$ and $b$ are relatively prime iff there exists integers $x$ and $y$ such that $1 = ax+by$.\\

If $a|c$ and  $b|c$, and $gcd(a,b) = 1$, then $ab|c$.\\

\textbf{Theorem 2.5 Euclid's Lemma:} If $a|bc$, and $gcd(a,b) = 1$, then $a|c$\\

\textbf{Theorem 2.6:} Let $a, b$ be integers, not both zero. For a positive integer $d$,  $d = gcd(a,d)$ iff\\

1. $d|a$ and $d|b$\\
2. Whenever $c|a$ and $c|b$, then $c|d$\\



\end{document}

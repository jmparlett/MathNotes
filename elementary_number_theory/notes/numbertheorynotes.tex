\documentclass[14pt]{extreport}
\usepackage{geometry} % see geometry.pdf on how to lay out the page. There's lots.
\usepackage{amsmath}
\usepackage{amssymb}
\usepackage{fancyhdr}
\usepackage{pgfplots}
\usepackage{setspace}
\pgfplotsset{compat=1.17} 
\geometry{a4paper} % or letter or a5paper or ... etc
% \geometry{landscape} % rotated page geometry

% See the ``Article customise'' template for come common customisations

\title{Elementary Number Theory Notes}
\author{Jonathan Parlett}



\begin{document}

\paragraph{1.0} \textbf{Introduction}\\\\

\textbf{Well Ordering Principle}: Every nonempty set $S$ of non-negative integers contains a least element; that is there is osme integer $a$ in $S$ such that $a \le b$ for all $b$'s belonging to $S$\\\\

\textbf{Theorem 1.1:} Archimedian property. If $a$ and $b$ are any positive integers, then there exists a positive integer $n$ such that $na \ge b$.\\\\

\textbf{Theorem 1.2} First Principle of Finite Induction. Let $S$ be the set of positive integers.\\
    (a) The integer 1 belongs to $S$\\
    (b) Whenever the integer $k$ is in $S$, the next integer $k+1$ must also be in $S$\\\\

\textbf{Theorem 1.2} Second Principle of Finite Induction. Let $S$ be the set of positive integers.\\
    (a) The integer 1 belongs to $S$\\
    (b') If $k$ is a positive integer such that $1, 2, \cdots ,k$ for $k \in S$, then $k+1$ must also be in $S$.\\\\


    Thus $S$ is the set of all positive integers.\\

\end{document}

\documentclass[14pt]{extreport}
\usepackage{geometry} % see geometry.pdf on how to lay out the page. There's lots.
\usepackage{amsmath}
\usepackage{amssymb}
\usepackage{fancyhdr}
\usepackage{pgfplots}
\usepackage{setspace}
\pgfplotsset{compat=1.17} 
\geometry{a4paper} % or letter or a5paper or ... etc
% \geometry{landscape} % rotated page geometry

% See the ``Article customise'' template for come common customisations

\title{Mafs tings}
\author{Jonathan Parlett}



\begin{document}
\textbf{Multiples in an Interval}\\
Given an interval $[l, h]$ and a whole number $a$ such that $a > 1$ and given that $r = a - (l\%a)$
there exist a multiple of $a$ on the interval if $l + r \le h$.\\\\

\textbf{If $a < l$}\\
then $r = a - l\%a$ which is the distance from $l$ to next multiple of $a$ then we can simply check if this is within the interval.

\textbf{If $a > l$}\\
then it suffices to check if $a < h$ and $l + r = l + a - l$ since $l\%a = l$ thus $l + r \le h$ simplifies to $a < h$.

This does not account for endpoints of the interval however so this formula only works for open intervals.

\textbf{If $a = l$}\\

\textbf{If $a = h$}\\
\end{document}
